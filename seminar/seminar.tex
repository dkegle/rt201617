\documentclass[12pt,a4paper]{amsart}
\usepackage[slovene]{babel}
\usepackage[utf8]{inputenc}
%\usepackage[T1]{fontenc}
\usepackage{amsmath,amssymb,amsfonts}
\usepackage[dvipsnames,usenames]{color}
\usepackage{algorithmicx,algpseudocode}

\textwidth 15cm
\textheight 24cm
\oddsidemargin.5cm
\evensidemargin.5cm
\topmargin-5mm
\addtolength{\footskip}{10pt}
\pagestyle{plain}

\overfullrule=15pt % oznaci predlogo vrstico


\newtheorem{definicija}{Definicija}[section]
\newtheorem{lema}[definicija]{Lema}
\newtheorem{izrek}[definicija]{Izrek}
\newtheorem{trditev}[definicija]{Trditev}
\newtheorem{posledica}[definicija]{Posledica}


\def\R{\mathbb R}
\def\N{\mathbb N}
\def\Z{\mathbb Z}
\def\C{\mathbb C}
\def\Q{\mathbb Q}


\begin{document}

\thispagestyle{empty}
\noindent{\large
University of Ljubljana \hfill Ljubljana, \today\\[1mm]
Faculty of Computer and Information Science  \\[5mm]
%IŠRM -- 2.~stopnja 
}
%\vfill
\begin{center}{\large
Computational topology\\[4mm]
% Seminarska naloga\\[4mm]
{\bf Text classification using persistent homology}\\[4mm]
Matija Čufar, Domen Keglevič\\[6mm]
}
\end{center}
\bigskip
% tu se zacne tekst seminarja
\section{Introduction}

\section{Algoritem}

\begin{definicija}
Funkcija $f\colon [a,b]\to\R$ je {\em zvezna}, "ce...
\end{definicija}
%
\begin{izrek}
Zvezna funkcija na zaprtem intervalu je enakomerno zvezna.
\end{izrek}
%
\proof
Izberimo $\varepsilon>0$.
\endproof

\subsection{{\em Fat line} in {\em fat curve}}

\subsection{Iskanje intervalov}

\subsection{Psevdo koda}

\begin{lema}
Naj bo $f$ zvezna in ...
\end{lema}

\section{Results}

% seznam uporabljene literature
\begin{thebibliography}{99}

\bibitem{oznaka-enote-za-sklic}
\textcolor{Red}{I.~Priimek, {\em Naslov "clanka}, okraj"sano ime revije {\bf letnik revije} (leto izida) strani od--do.}

\bibitem{navodilaOMF}
\textcolor{Red}{C.~Velkovrh, {\em Nekaj navodil avtorjem za pripravo rokopisa}, Obzornik mat.\ fiz.\ {\bf 21} (1974) 62--64.}

\end{thebibliography}

\end{document}

