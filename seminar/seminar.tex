\documentclass[12pt,a4paper]{amsart}
%\usepackage[slovene]{babel}
\usepackage[utf8]{inputenc}
%\usepackage[T1]{fontenc}
\usepackage{amsmath,amssymb,amsfonts}
\usepackage[dvipsnames,usenames]{color}
\usepackage{algorithmicx,algpseudocode}
\usepackage{graphicx}

\textwidth 15cm
\textheight 24cm
\oddsidemargin.5cm
\evensidemargin.5cm
\topmargin-5mm
\addtolength{\footskip}{10pt}
\pagestyle{plain}

\overfullrule=15pt % oznaci predlogo vrstico

\newtheorem{definition}{Definition}[section]
\newtheorem{lemma}[definition]{Lemma}
\newtheorem{theorem}[definition]{Theorem}
\newtheorem{corollary}[definition]{Corollary}

\def\R{\mathbb R}
\def\N{\mathbb N}
\def\Z{\mathbb Z}
\def\C{\mathbb C}
\def\Q{\mathbb Q}

\begin{document}

\thispagestyle{empty}
\noindent{\large
University of Ljubljana \hfill  \today\\[1mm]
Faculty of Computer and Information Science  \\[5mm]
%IŠRM -- 2.~stopnja
}
%\vfill
\begin{center}{\large
Computational topology\\[4mm]
% Seminarska naloga\\[4mm]
{\bf Text classification using persistent homology}\\[4mm]
Matija Čufar, Domen Keglevič\\[6mm]
}
\end{center}
\bigskip

\section{Introduction}

\emph{TODO: nekaj o vztrajni homologiji in mogoče analizi teksta}

In this report, we attempt to classify texts from four different domains by
comparing their persistence diagrams.

\section{Methods}

We have chosen to attempt to classify texts from the following domains:

\begin{itemize}
\item Excrepts from the Old and New Testaments of the Bible,
\item abstracts of articles from phys.org,
\item recipes from allrecipes.com.
\end{itemize}

For each of the domains, we picked ten texts, each at least 100 words long. We
used the Gudhi~\cite{maria2014gudhi}, a library for topological data analysis,
to compute persistent homology on the texts.

To compute the persistent homology of a data set, we first need to represent it
as a simplicial, or some other kind of complex. We used the following two
approaches to build simplicial complexes for each of the domains:

\subsection{Feature-based Alpha and Vietoris-Rips complexes}

Our first approach involved computing the following features for each of the
texts:

\begin{itemize}
  \item the ratio of (average word length)/(longest word length),
  \item the ratio of (average sentence length)/(longest sentence length),
  \item the ratio of the total number of three words with the highest tf-idf
    value among all the words,
  \item the ratio of the number of words of length $\le 8$ among all words,
  \item the ratio of the number of words of length $\ge 9$ among all words,
  \item the ratio of (number of different words)/(number of all words).
\end{itemize}

This gives us a point in $\R^6$ for each of the texts. We used these points to
build Alpha and Vietoris-Rips complexes on each of the domains.

\subsection{Distribtuion distance-based Vietoris-Rips complexes}

Our second approach involved computing the distributions of word and sentence
lengths and calculating the distances between the texts using the following
distance measures:

\begin{itemize}
\item The Hellinger distance:

\begin{equation*}
  H(P,Q) = \sqrt{\frac{1}{2} \sum_{i=1}^k\left(\sqrt{p_i} -
    \sqrt{q_i}\right)^2}\ ,
\end{equation*}

\item the Chi-squared distance:

\begin{equation*}
  \chi^2(P,Q) = \frac{(p_i - q_i)^2}{2(p_i + q_i + \varepsilon)}\ ,
\end{equation*}

\item the Euclidean distance:

\begin{equation*}
  E(P,Q) = \sqrt{\sum_{i=1}^k\left(p_i - q_i\right)^2}\ ,
\end{equation*}
\end{itemize}

\noindent
where $P$ and $Q$ are the discrete distributions and $p_i$ and $q_i$ are the
$i$-th bins of those distributions. The $\varepsilon$ in the Chi-squared
distance is a small constant used to avoid dividing by zero.

We used these distances to compute a distance matrix for each of the
domains and used the distance matrices to build Vetoris-Rips complexes.

\subsection{Domain comparsion}

When the simplicial complexes were built, we calculated persistence diagrams for
each complex and computed the bottleneck distances between them. We expected the
distances would help us distinguish between the texts. For example, the distance
between the texts taken from the Bible should be smaller than the rest.

\section{Results}

In the next secitons, we present the results given by both our approaches.

\subsection{Feature-based Alpha and Vietoris-Rips complexes} The bottleneck
distance matrix for the Alpha complex is shown in Table~\ref{tab:alpha}. The
barcode plots are shown in Figure~\ref{fig:barcode:alpha}.

As we can see from the distance matrix, the method did not produce meaningful
results. The first thing we notice is that all the distances are very small,
with the highest being only 0.008. Other examples that show the ineffectiveness
of this method are the fact that the Old Testament differs the most from the New
Testament and the fact that according to this method, physics article abstracts
are very similar to recipes.

Something we notice from the barcode plots is that
\emph{TODO: napisat neki o 4d zadevah v receptih}

We will not present the results for the Vietoris-Rips complex in this report,
because the results are very similar.

\begin{table}
  \centering
  \begin{tabular}{c|cccc}
                  & Old Testament & New Testament & phys.org & recipes \\ \hline
    Old Testament & 0.000 & 0.008 & 0.003 & 0.003 \\
    New Testament & 0.008 & 0.000 & 0.008 & 0.008 \\
    phys.org      & 0.003 & 0.008 & 0.000 & 0.002 \\
    recipes       & 0.003 & 0.008 & 0.002 & 0.000 \\
  \end{tabular}

  \caption{The distance matrix calculated from the Alpha complexes.}
  \label{tab:alpha}
\end{table}

\begin{figure}
  \centering
  \includegraphics[width=0.45\textwidth]{../plots/barcodes/bible-new-alpha}
  \includegraphics[width=0.45\textwidth]{../plots/barcodes/bible-old-alpha}
  \includegraphics[width=0.45\textwidth]{../plots/barcodes/phys-alpha}
  \includegraphics[width=0.45\textwidth]{../plots/barcodes/recipes-alpha}
  \caption{Persistence barcodes of the domains, calculated from the Alpha
    complexes.}
  \label{fig:barcode:alpha}
\end{figure}

\subsection{Distribtuion distance-based Vietoris-Rips complexes}

\emph{TODO: tekst}

\begin{table}
  \centering
  \begin{tabular}{c|cccc}
                  & Old Testament & New Testament & phys.org & recipes \\ \hline
    Old Testament & 0.000 & 0.015 & 0.040 & 0.112 \\
    New Testament & 0.015 & 0.000 & 0.033 & 0.103 \\
    phys.org      & 0.040 & 0.033 & 0.000 & 0.105 \\
    recipes       & 0.112 & 0.103 & 0.105 & 0.000 \\
  \end{tabular}

  \caption{The distance matrix calculated from the Vietoris-Rips complexes,
    built using the Hellinger distance between sentence length distribution.}
  \label{tab:hell}
\end{table}

\begin{figure}
  \centering
  \includegraphics[width=0.45\textwidth]{../plots/barcodes/bible-new-hell}
  \includegraphics[width=0.45\textwidth]{../plots/barcodes/bible-old-hell}
  \includegraphics[width=0.45\textwidth]{../plots/barcodes/phys-hell}
  \includegraphics[width=0.45\textwidth]{../plots/barcodes/recipes-hell}
  \caption{Persistence barcodes of the domains, calculated from the
    Vietoris-Rips complexes, built using the Hellinger distance between sentence
    length distributions.}
  \label{fig:barcode:hell}
\end{figure}

\section{Conclusion}

As we saw in the previous section, the results are not very good.



\section{Authors contributions}

% biblio
\bibliographystyle{plain}
\bibliography{biblio}


\end{document}
